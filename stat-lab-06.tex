% Options for packages loaded elsewhere
\PassOptionsToPackage{unicode}{hyperref}
\PassOptionsToPackage{hyphens}{url}
%
\documentclass[
]{article}
\usepackage{amsmath,amssymb}
\usepackage{lmodern}
\usepackage{iftex}
\ifPDFTeX
  \usepackage[T1]{fontenc}
  \usepackage[utf8]{inputenc}
  \usepackage{textcomp} % provide euro and other symbols
\else % if luatex or xetex
  \usepackage{unicode-math}
  \defaultfontfeatures{Scale=MatchLowercase}
  \defaultfontfeatures[\rmfamily]{Ligatures=TeX,Scale=1}
\fi
% Use upquote if available, for straight quotes in verbatim environments
\IfFileExists{upquote.sty}{\usepackage{upquote}}{}
\IfFileExists{microtype.sty}{% use microtype if available
  \usepackage[]{microtype}
  \UseMicrotypeSet[protrusion]{basicmath} % disable protrusion for tt fonts
}{}
\makeatletter
\@ifundefined{KOMAClassName}{% if non-KOMA class
  \IfFileExists{parskip.sty}{%
    \usepackage{parskip}
  }{% else
    \setlength{\parindent}{0pt}
    \setlength{\parskip}{6pt plus 2pt minus 1pt}}
}{% if KOMA class
  \KOMAoptions{parskip=half}}
\makeatother
\usepackage{xcolor}
\usepackage[margin=1in]{geometry}
\usepackage{color}
\usepackage{fancyvrb}
\newcommand{\VerbBar}{|}
\newcommand{\VERB}{\Verb[commandchars=\\\{\}]}
\DefineVerbatimEnvironment{Highlighting}{Verbatim}{commandchars=\\\{\}}
% Add ',fontsize=\small' for more characters per line
\usepackage{framed}
\definecolor{shadecolor}{RGB}{248,248,248}
\newenvironment{Shaded}{\begin{snugshade}}{\end{snugshade}}
\newcommand{\AlertTok}[1]{\textcolor[rgb]{0.94,0.16,0.16}{#1}}
\newcommand{\AnnotationTok}[1]{\textcolor[rgb]{0.56,0.35,0.01}{\textbf{\textit{#1}}}}
\newcommand{\AttributeTok}[1]{\textcolor[rgb]{0.77,0.63,0.00}{#1}}
\newcommand{\BaseNTok}[1]{\textcolor[rgb]{0.00,0.00,0.81}{#1}}
\newcommand{\BuiltInTok}[1]{#1}
\newcommand{\CharTok}[1]{\textcolor[rgb]{0.31,0.60,0.02}{#1}}
\newcommand{\CommentTok}[1]{\textcolor[rgb]{0.56,0.35,0.01}{\textit{#1}}}
\newcommand{\CommentVarTok}[1]{\textcolor[rgb]{0.56,0.35,0.01}{\textbf{\textit{#1}}}}
\newcommand{\ConstantTok}[1]{\textcolor[rgb]{0.00,0.00,0.00}{#1}}
\newcommand{\ControlFlowTok}[1]{\textcolor[rgb]{0.13,0.29,0.53}{\textbf{#1}}}
\newcommand{\DataTypeTok}[1]{\textcolor[rgb]{0.13,0.29,0.53}{#1}}
\newcommand{\DecValTok}[1]{\textcolor[rgb]{0.00,0.00,0.81}{#1}}
\newcommand{\DocumentationTok}[1]{\textcolor[rgb]{0.56,0.35,0.01}{\textbf{\textit{#1}}}}
\newcommand{\ErrorTok}[1]{\textcolor[rgb]{0.64,0.00,0.00}{\textbf{#1}}}
\newcommand{\ExtensionTok}[1]{#1}
\newcommand{\FloatTok}[1]{\textcolor[rgb]{0.00,0.00,0.81}{#1}}
\newcommand{\FunctionTok}[1]{\textcolor[rgb]{0.00,0.00,0.00}{#1}}
\newcommand{\ImportTok}[1]{#1}
\newcommand{\InformationTok}[1]{\textcolor[rgb]{0.56,0.35,0.01}{\textbf{\textit{#1}}}}
\newcommand{\KeywordTok}[1]{\textcolor[rgb]{0.13,0.29,0.53}{\textbf{#1}}}
\newcommand{\NormalTok}[1]{#1}
\newcommand{\OperatorTok}[1]{\textcolor[rgb]{0.81,0.36,0.00}{\textbf{#1}}}
\newcommand{\OtherTok}[1]{\textcolor[rgb]{0.56,0.35,0.01}{#1}}
\newcommand{\PreprocessorTok}[1]{\textcolor[rgb]{0.56,0.35,0.01}{\textit{#1}}}
\newcommand{\RegionMarkerTok}[1]{#1}
\newcommand{\SpecialCharTok}[1]{\textcolor[rgb]{0.00,0.00,0.00}{#1}}
\newcommand{\SpecialStringTok}[1]{\textcolor[rgb]{0.31,0.60,0.02}{#1}}
\newcommand{\StringTok}[1]{\textcolor[rgb]{0.31,0.60,0.02}{#1}}
\newcommand{\VariableTok}[1]{\textcolor[rgb]{0.00,0.00,0.00}{#1}}
\newcommand{\VerbatimStringTok}[1]{\textcolor[rgb]{0.31,0.60,0.02}{#1}}
\newcommand{\WarningTok}[1]{\textcolor[rgb]{0.56,0.35,0.01}{\textbf{\textit{#1}}}}
\usepackage{graphicx}
\makeatletter
\def\maxwidth{\ifdim\Gin@nat@width>\linewidth\linewidth\else\Gin@nat@width\fi}
\def\maxheight{\ifdim\Gin@nat@height>\textheight\textheight\else\Gin@nat@height\fi}
\makeatother
% Scale images if necessary, so that they will not overflow the page
% margins by default, and it is still possible to overwrite the defaults
% using explicit options in \includegraphics[width, height, ...]{}
\setkeys{Gin}{width=\maxwidth,height=\maxheight,keepaspectratio}
% Set default figure placement to htbp
\makeatletter
\def\fps@figure{htbp}
\makeatother
\setlength{\emergencystretch}{3em} % prevent overfull lines
\providecommand{\tightlist}{%
  \setlength{\itemsep}{0pt}\setlength{\parskip}{0pt}}
\setcounter{secnumdepth}{-\maxdimen} % remove section numbering
\ifLuaTeX
  \usepackage{selnolig}  % disable illegal ligatures
\fi
\IfFileExists{bookmark.sty}{\usepackage{bookmark}}{\usepackage{hyperref}}
\IfFileExists{xurl.sty}{\usepackage{xurl}}{} % add URL line breaks if available
\urlstyle{same} % disable monospaced font for URLs
\hypersetup{
  pdftitle={Lab 06 (stat)},
  pdfauthor={Moushreeta Debroy(22122132)},
  hidelinks,
  pdfcreator={LaTeX via pandoc}}

\title{Lab 06 (stat)}
\author{Moushreeta Debroy(22122132)}
\date{2022-11-30}

\begin{document}
\maketitle

Question 1:To test the claim that the resistance of an electric wire can
be reduced by more than 0.050 ohm by alloying,32 values obtained for
standard wire yielded π = 0.136ohm (sample mean) and sd1 = 0.004 ohm and
32 values obtained for alloyed wire yielded y ̄ = 0.083 ohm and sd2 =
0.005 ohm. At the 0.05 level of significance, does this support the
claim? Evaluate the 90\% confidence interval of difference between
means? {[}Hint: generate the two sample using rnorm function{]}

H0:mu1-mu2=0.05 H1:mu1-mu2!=0.05

\begin{Shaded}
\begin{Highlighting}[]
\NormalTok{x}\OtherTok{\textless{}{-}}\FunctionTok{rnorm}\NormalTok{(}\DecValTok{32}\NormalTok{,}\AttributeTok{mean=}\FloatTok{0.136}\NormalTok{,}\AttributeTok{sd=}\FloatTok{0.004}\NormalTok{)}
\NormalTok{y}\OtherTok{\textless{}{-}}\FunctionTok{rnorm}\NormalTok{(}\DecValTok{32}\NormalTok{,}\AttributeTok{mean=}\FloatTok{0.083}\NormalTok{,}\AttributeTok{sd=}\FloatTok{0.005}\NormalTok{)}
\end{Highlighting}
\end{Shaded}

\begin{Shaded}
\begin{Highlighting}[]
\NormalTok{?t.test}
\end{Highlighting}
\end{Shaded}

\begin{verbatim}
## starting httpd help server ... done
\end{verbatim}

\begin{Shaded}
\begin{Highlighting}[]
\FunctionTok{t.test}\NormalTok{(x,y,}\AttributeTok{mu=}\FloatTok{0.05}\NormalTok{,}\AttributeTok{alternative =} \StringTok{"two.sided"}\NormalTok{,}\AttributeTok{var.equal =}\NormalTok{ F)}
\end{Highlighting}
\end{Shaded}

\begin{verbatim}
## 
##  Welch Two Sample t-test
## 
## data:  x and y
## t = 0.74081, df = 59.938, p-value = 0.4617
## alternative hypothesis: true difference in means is not equal to 0.05
## 95 percent confidence interval:
##  0.04870340 0.05282183
## sample estimates:
##  mean of x  mean of y 
## 0.13520044 0.08443783
\end{verbatim}

Question 2: Given the marks of 30 students for a test conducted before
and after an online course. Test whether the course was effective or
not.

\begin{Shaded}
\begin{Highlighting}[]
\FunctionTok{library}\NormalTok{(readxl)}
\NormalTok{marks\_data }\OtherTok{\textless{}{-}} \FunctionTok{read\_excel}\NormalTok{(}\StringTok{"C:/Users/PRASANTA/Downloads/marks data.xlsx"}\NormalTok{)}
\FunctionTok{View}\NormalTok{(marks\_data)}
\end{Highlighting}
\end{Shaded}

\begin{Shaded}
\begin{Highlighting}[]
\FunctionTok{colnames}\NormalTok{(marks\_data)}
\end{Highlighting}
\end{Shaded}

\begin{verbatim}
## [1] "Test 1" "Test 2"
\end{verbatim}

\begin{Shaded}
\begin{Highlighting}[]
\NormalTok{x}\OtherTok{\textless{}{-}}\NormalTok{marks\_data}\SpecialCharTok{$}\StringTok{\textasciigrave{}}\AttributeTok{Test 1}\StringTok{\textasciigrave{}}
\NormalTok{x}
\end{Highlighting}
\end{Shaded}

\begin{verbatim}
##  [1]  8.0  5.0  8.0  7.0  4.0  4.0  5.0  3.0 10.0  6.0  5.0  5.0  2.0  3.0  5.0
## [16]  4.0  5.0  5.0  2.5  5.0  6.5  4.0  6.5  3.0  5.0  1.0  5.0  6.0  4.0
\end{verbatim}

\begin{Shaded}
\begin{Highlighting}[]
\NormalTok{y}\OtherTok{\textless{}{-}}\NormalTok{marks\_data}\SpecialCharTok{$}\StringTok{\textasciigrave{}}\AttributeTok{Test 2}\StringTok{\textasciigrave{}}
\NormalTok{y}
\end{Highlighting}
\end{Shaded}

\begin{verbatim}
##  [1] 4 7 6 7 6 6 7 3 7 7 7 7 6 4 6 6 7 5 7 5 3 3 6 5 6 5 5 6 5
\end{verbatim}

d=difference of marks of students before and after taking the course
H0:mean(d)=0 H1:mean(d)!0

\begin{Shaded}
\begin{Highlighting}[]
\NormalTok{x}\OtherTok{=}\NormalTok{marks\_data}\SpecialCharTok{$}\StringTok{\textasciigrave{}}\AttributeTok{Test 1}\StringTok{\textasciigrave{}}
\NormalTok{y}\OtherTok{=}\NormalTok{marks\_data}\SpecialCharTok{$}\StringTok{\textasciigrave{}}\AttributeTok{Test 2}\StringTok{\textasciigrave{}}
\FunctionTok{t.test}\NormalTok{(x,y,}\AttributeTok{paired=}\ConstantTok{TRUE}\NormalTok{,}\AttributeTok{alternative =} \StringTok{"two.sided"}\NormalTok{,}\AttributeTok{conf.level =} \FloatTok{0.95}\NormalTok{)}
\end{Highlighting}
\end{Shaded}

\begin{verbatim}
## 
##  Paired t-test
## 
## data:  x and y
## t = -1.9301, df = 28, p-value = 0.06379
## alternative hypothesis: true mean difference is not equal to 0
## 95 percent confidence interval:
##  -1.52821749  0.04545887
## sample estimates:
## mean difference 
##      -0.7413793
\end{verbatim}

\end{document}
